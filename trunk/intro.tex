\begin{abstract}
With conventional memory technologies approaching their scaling limit, the
search for a new technology has gained increased attention in the recent years.
Resistive RAM (ReRAM), with its superior write latency and energy, small cell
size ($4F^2$ for a single level), and support for 3D stacking, has been a
promising candidate among emerging memory technologies.  A key advantage of
ReRAM comes from its non-linear nature, which enables a cross-point array
structure without having a dedicated access transistor for each cell. 

While the cross-point design is effective in improving the memory density, it has
inherent disadvantages which introduce extra design challenges. Based on the
device characteristics, we perform a comprehensive analysis of issues related
to reliability, energy consumption, area overhead, and performance for the
cross-point arrays. In addition to the cell-level analysis, we discuss
different programming schemes specifically suited for cross-point arrays.  We
then study the area, energy, and bandwidth of a 256 Mbits ReRAM macro in detail
for various write schemes.  The simulation results enable designers to identify
the most performance/energy/area efficient ReRAM organization and cell
parameters that meet specific design goals early in the design stage.
\end{abstract}

\vspace{-15pt}
\section{Introduction}\label{sec:intro}
The scaling of traditional memory technologies, such as DRAM and FLASH, is
approaching its physical limit. In the past few years, emerging
non-volatile memory technologies~(NVM), such as Phase Change RAM~(PCRAM),
Spin-transfer-torque RAM~(STT-RAM), and Resistive RAM~(ReRAM) have been
widely studied as potential candidates for the next generation memory
technologies to meet the requirement of higher density, faster access
time, and lower power consumption. Among all of these emerging memory
technologies, ReRAM has many unique characteristics, including simple
structure, nonlinearity, and high resistance ratio, making itself one of
the most promising technologies. Researchers have shown that the
state-of-the-art single-level-cell ReRAM can achieve $7.2ns$ random access
time for both read and write operations with a resistance ratio larger
than 100~\cite{ReRAM_ISSCC2011_Sheu}. Also, HP labs and Hynix have already
announced plans to commercialize memristor-based ReRAM and predicted that
ReRAM could eventually replace traditional memory
technologies~\cite{memristor:HpHynix}.

Unlike other non-volatile memory technologies, ReRAM can be implemented in
a cross-point style structure without any access
device~\cite{crossbar_unity,crossbar_Panasonic}. Specifically, in a nano
cross-point array, each bistable ReRAM cell is sandwiched by two
orthogonal nanowires. Thus the area occupied by each cell is $4F^2$ per
bit. However, the simplicity of the access-device-free, cross-point
structure introduces challenges to the peripheral circuit and memory
organization design. While there have been prior studies on cross-point
ReRAM
arrays~\cite{crossbar_NANO2002_Ziegler,crossbar_NANO08_Flocke,crossbar_TED_2010,crossbar_NANO2003_Ziegler,2011_sb},
they do not consider the effect of voltage drivers and programming methods
on the array. In addition, detailed area, energy, and performance analysis
is also absent. In this work, we address the design challenges of
cross-point structure based ReRAM. We use a mathematical model to evaluate
memory reliability, energy consumption, and area overhead for different
designs and cell parameters. The advantages of nonlinearity $K_r$ and
write current $I_w$ scaling are all discussed in detail. In addition, the
simulation results of area, energy, and write throughput trade-offs are
presented. Our study allows for exploring the most energy/area efficient
ReRAM design with different design constraints and cell parameters at the
very beginning of the design stage. Moreover, system designers can also
leverage the proposed model to provide valuable feedback to device
researchers who will in turn adjust ReRAM cell design. We believe that
this kind of collaboration will be very helpful to shorten the time to
market of ReRAM memory.

%The rest of this paper is organized as follows. In
%Section~\ref{sec:preliminary}, an overview of ReRAM technology and
%cross-point architectures is given. Section~\ref{sec:model} % Section~III
%discusses the proposed mathematical model for the cross-point structure
%ReRAM and the edge conditions for different write and read schemes.
%Section~\ref{sec:w_and_r} analyzes different design constraints of write
%and read operations on cross-point based ReRAM arrays. The energy
%consumption and area overheads are also analyzed in this section. Then in
%Section~\ref{sec:scale}, the effect of nonlinearity and write current on
%the design constraints is evaluated. Finally, the conclusion is presented
%in Section~\ref{sec:conclusion}.
