\vspace{-5pt}
\section{Conclusion}\label{sec:conclusion}
%ReRAM is a promising candidate for next-generation non-volatile memory
%technology. The area efficient cross-point structure is the most
%attractive memory organization for ReRAM memories. However, problems
%inherent in the cross-point structure, such as the existence of sneak
%current and voltage drops along the wires introduce challenges to the
%design of reliable ReRAM cross-point memory arrays.
In this paper, we use a mathematical model to study in detail how reliability
affects the array organization, size, energy consumption, and area overheads of
cross-point arrays.  The size of a cross-point is limited by the peripheral
circuit overhead and sneak current.  Our simulation results show that with best
possible device non-linearity and drive current, the maximum array size cannot
exceed 1024x1024 without compromising reliability.  We also showed that
multi-bit writes is more energy efficient than single-bit write, however, the
latter significantly reduces the complexity of peripheral circuits and provides
better area efficiency.  Both high nonlinearity and low write current are key
to reduce energy and area of cross-point arrays.  Finally, since memory
bandwidth is an important design constraint, we studied various designs that
maximizes bandwidth for a given area and energy budget.  Through our case
study, we show that there is an optimal word size for a given device parameter
that has the best energy, area, and bandwidth properties. 
%  According to our macro-level
%analysis, we figure out that we have to either sacrifice the area
%efficiency or increase the energy budget to improve the bandwidth of the
%ReRAM macro. In addition, for different optimization target (bandwidth per
%nanojoule, bandwidth per square millimeter, and bandwidth per nanojoule),
%a optimal number of bits per write exists.
