\vspace{-5pt}
\section{Conclusion}\label{sec:conclusion}
ReRAM is a promising candidate for next-generation non-volatile memory
technology. The area efficient cross-point structure is the most
attractive memory organization for ReRAM memories. However, problems
inherent in the cross-point structure, such as the existence of sneak
current and voltage drops along the wires introduce challenges to the
design of reliable ReRAM cross-point memory arrays. In this paper, we use
a mathematical model to study in detail how reliability affects the array
organization, size, energy consumption, and area overheads of cross-point
arrays. The simulation results show that multi-bit write operation is more
energy efficient than single-bit write operation, and therefore is more
suitable for energy-constrained design. However, for an area-constrained
design, single-bit write operation is better. Besides, we point out that
both increasing nonlinearity and scaling of write current of the ReRAM
cell can reduce the energy consumption and area overhead significantly,
and it is favorable for large, energy efficient ReRAM design. According to
our macro-level analysis, we figure out that we have to either sacrifice
the area efficiency or increase the energy budget to improve the bandwidth
of the ReRAM macro.
